% Seccion: Resumen Estadistico
% Generado automaticamente por HidroPluvial

\section{Resumen Estadístico}

\begin{table}[H]
\centering
\begin{tabular}{lr}
\toprule
Estadístico & Valor \\
\midrule
Número de análisis & 20 \\
Caudal pico máximo & 16.541 m$^3$/s (kirpich + gz $T_r$=25) \\
Caudal pico mínimo & 6.503 m$^3$/s (desbordes + gz $T_r$=2) \\
Caudal pico promedio & 11.633 m$^3$/s \\
Variación máx/mín & 154.3\% \\
\bottomrule
\end{tabular}
\caption{Resumen estadístico de caudales pico}
\label{tab:summary}
\end{table}

\subsection{Observaciones}

\begin{itemize}
    \item La variación entre métodos es del 154.3\%, lo que indica alta sensibilidad a la metodología empleada.
    \item El caudal de diseño recomendado depende del nivel de riesgo aceptable para la obra.
    \item Para obras de infraestructura crítica, considerar el valor máximo.
    \item Para drenaje menor, puede utilizarse el valor promedio o el correspondiente al $T_r$ de diseño.
\end{itemize}
