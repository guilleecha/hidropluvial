\documentclass[11pt, a4paper]{article}

% Paquetes básicos
\usepackage[utf8]{inputenc}
\usepackage[spanish]{babel}
\usepackage[margin=2.5cm]{geometry}
\usepackage{booktabs}
\usepackage{siunitx}
\usepackage{amsmath}
\usepackage{float}
\usepackage{graphicx}
\usepackage{hyperref}
\usepackage{tikz}
\usepackage{pgfplots}
\pgfplotsset{compat=1.18}

% Metadatos
\title{Memoria de Cálculo: Curvas IDF}
\author{\VAR{author | escape_latex}}
\date{\VAR{date}}

\begin{document}

\maketitle

\section{Datos de Entrada}

\begin{table}[H]
\centering
\begin{tabular}{lr}
\toprule
Parámetro & Valor \\
\midrule
Método & \VAR{method} \\
$P_{3,10}$ base & \VAR{p3_10} mm \\
\BLOCK{if area}
Área de cuenca & \VAR{area} km² \\
\BLOCK{endif}
\bottomrule
\end{tabular}
\caption{Parámetros de entrada}
\label{tab:input}
\end{table}

\section{Tabla de Intensidades}

\VAR{idf_table}

\section{Fórmulas Utilizadas}

\subsection{Factor por Período de Retorno (CT)}

\begin{equation}
C_T(T_r) = 0.5786 - 0.4312 \times \log_{10}\left[\ln\left(\frac{T_r}{T_r - 1}\right)\right]
\end{equation}

\subsection{Factor por Área de Cuenca (CA)}

\begin{equation}
C_A(A_c, d) = 1.0 - 0.3549 \times d^{-0.4272} \times \left(1.0 - e^{-0.005792 \times A_c}\right)
\end{equation}

\subsection{Intensidad}

Para $d < 3$ horas:
\begin{equation}
I(d) = \frac{P_{3,10} \times C_T(T_r) \times C_A \times 0.6208}{(d + 0.0137)^{0.5639}}
\end{equation}

Para $d \geq 3$ horas:
\begin{equation}
I(d) = \frac{P_{3,10} \times C_T(T_r) \times C_A \times 1.0287}{(d + 1.0293)^{0.8083}}
\end{equation}

\vfill
\noindent\rule{\textwidth}{0.4pt}
\footnotesize
Generado por HidroPluvial v\VAR{version} el \VAR{date}.

\end{document}
