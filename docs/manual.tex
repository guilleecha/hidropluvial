% ============================================================================
% HIDROPLUVIAL - Manual de Usuario
% Herramienta de Cálculos Hidrológicos
% ============================================================================

\documentclass[11pt, a4paper, twoside]{report}

% ============================================================================
% PAQUETES
% ============================================================================

% Codificación y lenguaje
\usepackage[utf8]{inputenc}
\usepackage[T1]{fontenc}
\usepackage[spanish]{babel}

% Geometría
\usepackage[
    top=2.5cm,
    bottom=2.5cm,
    left=3cm,
    right=2.5cm,
    headheight=14pt
]{geometry}

% Tipografía
\usepackage{lmodern}
\usepackage{microtype}

% Matemáticas
\usepackage{amsmath}
\usepackage{amssymb}
\usepackage{siunitx}
\sisetup{
    output-decimal-marker = {,},
    group-separator = {.},
    group-minimum-digits = 4
}

% Tablas
\usepackage{booktabs}
\usepackage{longtable}
\usepackage{array}
\usepackage{multirow}

% Gráficos
\usepackage{graphicx}
\usepackage{tikz}
\usepackage{pgfplots}
\pgfplotsset{compat=1.18}

% Código
\usepackage{listings}
\usepackage{xcolor}

% Configuración de listings para Python
\definecolor{codegreen}{rgb}{0,0.6,0}
\definecolor{codegray}{rgb}{0.5,0.5,0.5}
\definecolor{codepurple}{rgb}{0.58,0,0.82}
\definecolor{backcolour}{rgb}{0.97,0.97,0.97}

\lstdefinestyle{pythonstyle}{
    backgroundcolor=\color{backcolour},
    commentstyle=\color{codegreen},
    keywordstyle=\color{blue},
    numberstyle=\tiny\color{codegray},
    stringstyle=\color{codepurple},
    basicstyle=\ttfamily\small,
    breakatwhitespace=false,
    breaklines=true,
    captionpos=b,
    keepspaces=true,
    numbers=left,
    numbersep=5pt,
    showspaces=false,
    showstringspaces=false,
    showtabs=false,
    tabsize=4,
    frame=single,
    rulecolor=\color{gray!30}
}

\lstdefinestyle{bashstyle}{
    backgroundcolor=\color{black!5},
    basicstyle=\ttfamily\small,
    breaklines=true,
    frame=single,
    rulecolor=\color{gray!30}
}

\lstset{style=pythonstyle}

% Enlaces
\usepackage{hyperref}
\hypersetup{
    colorlinks=true,
    linkcolor=blue!70!black,
    urlcolor=blue!70!black,
    citecolor=green!50!black
}

% Encabezados y pies
\usepackage{fancyhdr}
\pagestyle{fancy}
\fancyhf{}
\fancyhead[LE,RO]{\thepage}
\fancyhead[LO]{\nouppercase{\rightmark}}
\fancyhead[RE]{\nouppercase{\leftmark}}
\renewcommand{\headrulewidth}{0.4pt}

% Otros
\usepackage{float}
\usepackage{enumitem}
\usepackage{caption}

% ============================================================================
% COMANDOS PERSONALIZADOS
% ============================================================================

\newcommand{\hidropluvial}{\texttt{hidropluvial}}
\newcommand{\comando}[1]{\texttt{#1}}
\newcommand{\archivo}[1]{\texttt{#1}}

% ============================================================================
% INFORMACIÓN DEL DOCUMENTO
% ============================================================================

\title{
    \vspace{-2cm}
    \Huge\textbf{HidroPluvial} \\[0.5cm]
    \Large Manual de Usuario \\[1cm]
    \large Herramienta de Cálculos Hidrológicos \\
    con Generación de Reportes \LaTeX
}

\author{
    \large Versión 0.1.0 \\[0.5cm]
    \normalsize Diciembre 2024
}

\date{}

% ============================================================================
% DOCUMENTO
% ============================================================================

\begin{document}

% Portada
\maketitle
\thispagestyle{empty}

\vfill

\begin{center}
\begin{tabular}{ll}
\textbf{Repositorio:} & \url{https://github.com/guilleecha/hidropluvial} \\
\textbf{Licencia:} & MIT \\
\textbf{Python:} & 3.11+ \\
\end{tabular}
\end{center}

\newpage

% Índice
\tableofcontents
\newpage

% ============================================================================
% CAPÍTULO 1: INTRODUCCIÓN
% ============================================================================

\chapter{Introducción}

\section{Descripción General}

\hidropluvial{} es una herramienta Python para cálculos hidrológicos orientada principalmente a Uruguay, con soporte completo para la metodología DINAGUA. Permite realizar análisis de curvas IDF, generar hietogramas de diseño, calcular escorrentía y generar reportes técnicos en formato \LaTeX.

\section{Características Principales}

\begin{itemize}
    \item \textbf{Curvas IDF:} Método DINAGUA Uruguay con factores $C_T$ y $C_A$, además de métodos internacionales (Sherman, Bernard, Koutsoyiannis).

    \item \textbf{Distribuciones Temporales:} Bloques alternantes, método Chicago, SCS Tipo I/IA/II/III, curvas de Huff, tormentas bimodales.

    \item \textbf{Tiempo de Concentración:} Fórmulas de Kirpich, NRCS, Témez, California Culverts, FAA.

    \item \textbf{Escorrentía:} Método SCS Curve Number con ajuste por AMC, método Racional con factor $C_f$.

    \item \textbf{Hidrogramas:} SCS triangular y curvilíneo, Snyder, Clark, convolución.

    \item \textbf{Reportes:} Generación automática de memorias de cálculo en \LaTeX{} con gráficos TikZ/PGFPlots.

    \item \textbf{Exportación:} CSV, JSON, figuras TikZ standalone.
\end{itemize}

\section{Instalación}

\subsection{Requisitos}

\begin{itemize}
    \item Python 3.11 o superior
    \item pip (gestor de paquetes Python)
    \item Git (opcional, para clonar el repositorio)
\end{itemize}

\subsection{Instalación desde Repositorio}

\begin{lstlisting}[style=bashstyle, language=bash]
# Clonar repositorio
git clone https://github.com/guilleecha/hidropluvial.git
cd hidropluvial

# Crear entorno virtual
python -m venv .venv

# Activar entorno virtual
# Windows:
.venv\Scripts\activate
# Linux/Mac:
source .venv/bin/activate

# Instalar en modo desarrollo
pip install -e .

# Verificar instalacion
python -m hidropluvial --help
\end{lstlisting}

% ============================================================================
% CAPÍTULO 2: CURVAS IDF
% ============================================================================

\chapter{Curvas Intensidad-Duración-Frecuencia}

\section{Método DINAGUA Uruguay}

El método DINAGUA es el estándar para el diseño hidrológico en Uruguay. Se basa en el trabajo de Rodríguez Fontal (1980) y utiliza el parámetro $P_{3,10}$ (precipitación de 3 horas para un período de retorno de 10 años) como base para la regionalización.

\subsection{Factor por Período de Retorno ($C_T$)}

El factor $C_T$ ajusta la intensidad según el período de retorno deseado:

\begin{equation}
C_T(T_r) = 0{,}5786 - 0{,}4312 \times \log_{10}\left[\ln\left(\frac{T_r}{T_r - 1}\right)\right]
\label{eq:ct}
\end{equation}

\begin{table}[H]
\centering
\caption{Valores de $C_T$ para distintos períodos de retorno}
\label{tab:ct}
\begin{tabular}{ccccccc}
\toprule
$T_r$ (años) & 2 & 5 & 10 & 25 & 50 & 100 \\
\midrule
$C_T$ & 0,647 & 0,860 & 1,000 & 1,178 & 1,309 & 1,440 \\
\bottomrule
\end{tabular}
\end{table}

\subsection{Factor por Área de Cuenca ($C_A$)}

Para cuencas con área mayor a 1 km², se aplica una reducción de la intensidad:

\begin{equation}
C_A(A_c, d) = 1{,}0 - 0{,}3549 \times d^{-0{,}4272} \times \left(1{,}0 - e^{-0{,}005792 \times A_c}\right)
\label{eq:ca}
\end{equation}

donde:
\begin{itemize}
    \item $A_c$: área de la cuenca en km²
    \item $d$: duración de la tormenta en horas
\end{itemize}

\textbf{Nota:} Para $A_c \leq 1$ km², $C_A = 1{,}0$.

\subsection{Ecuaciones de Intensidad}

\textbf{Para duraciones menores a 3 horas:}
\begin{equation}
I(d) = \frac{P_{3,10} \times C_T(T_r) \times C_A \times 0{,}6208}{(d + 0{,}0137)^{0{,}5639}}
\label{eq:i_short}
\end{equation}

\textbf{Para duraciones de 3 horas o mayores:}
\begin{equation}
I(d) = \frac{P_{3,10} \times C_T(T_r) \times C_A \times 1{,}0287}{(d + 1{,}0293)^{0{,}8083}}
\label{eq:i_long}
\end{equation}

donde $I$ está en mm/hr y $d$ en horas.

\subsection{Valores de $P_{3,10}$ por Departamento}

\begin{table}[H]
\centering
\caption{Valores de $P_{3,10}$ por departamento de Uruguay}
\label{tab:p310}
\begin{tabular}{lclc}
\toprule
Departamento & $P_{3,10}$ (mm) & Departamento & $P_{3,10}$ (mm) \\
\midrule
Artigas & 95 & Maldonado & 83 \\
Canelones & 80 & Montevideo & 78 \\
Cerro Largo & 95 & Paysandú & 90 \\
Colonia & 78 & Río Negro & 88 \\
Durazno & 88 & Rivera & 95 \\
Flores & 85 & Rocha & 85 \\
Florida & 85 & Salto & 92 \\
Lavalleja & 85 & San José & 80 \\
& & Soriano & 85 \\
& & Tacuarembó & 92 \\
& & Treinta y Tres & 90 \\
\bottomrule
\end{tabular}
\end{table}

\section{Uso desde CLI}

\begin{lstlisting}[style=bashstyle, language=bash]
# Intensidad puntual: Montevideo, 3 horas, Tr=25 anos
python -m hidropluvial idf uruguay 78 3 --tr 25

# Con correccion por area (50 km2)
python -m hidropluvial idf uruguay 78 3 --tr 25 --area 50

# Tabla IDF completa
python -m hidropluvial idf tabla-uy 78

# Ver valores P3,10 por departamento
python -m hidropluvial idf departamentos
\end{lstlisting}

\section{Uso desde Python}

\begin{lstlisting}[language=Python]
from hidropluvial.core import dinagua_intensity, generate_dinagua_idf_table

# Calcular intensidad puntual
result = dinagua_intensity(
    p3_10=78,           # mm
    return_period_yr=25,
    duration_hr=3,
    area_km2=None       # sin correccion por area
)

print(f"CT: {result.ct:.4f}")
print(f"CA: {result.ca:.4f}")
print(f"Intensidad: {result.intensity_mmhr:.2f} mm/hr")
print(f"Precipitacion: {result.depth_mm:.2f} mm")

# Generar tabla completa
tabla = generate_dinagua_idf_table(p3_10=78, area_km2=50)
\end{lstlisting}

% ============================================================================
% CAPÍTULO 3: DISTRIBUCIONES TEMPORALES
% ============================================================================

\chapter{Distribuciones Temporales de Lluvia}

\section{Método de Bloques Alternantes}

El método de bloques alternantes genera un hietograma sintético a partir de una curva IDF. Los pasos son:

\begin{enumerate}
    \item Calcular profundidades acumuladas desde la curva IDF: $P(d) = I(d) \times d$
    \item Computar incrementos: $\Delta P_n = P(n \cdot \Delta t) - P((n-1) \cdot \Delta t)$
    \item Ordenar incrementos de mayor a menor
    \item Distribuir alternadamente alrededor del pico (centro por defecto)
\end{enumerate}

\subsection{Uso desde CLI}

\begin{lstlisting}[style=bashstyle, language=bash]
# Hietograma DINAGUA: P3,10=78mm, 3 horas, Tr=25, dt=5min
python -m hidropluvial storm uruguay 78 3 --tr 25 --dt 5

# Exportar a CSV
python -m hidropluvial export storm-csv 78 3 --tr 25 -o storm.csv
\end{lstlisting}

\subsection{Uso desde Python}

\begin{lstlisting}[language=Python]
from hidropluvial.core import alternating_blocks_dinagua

hietograma = alternating_blocks_dinagua(
    p3_10=78,
    return_period_yr=25,
    duration_hr=3,
    dt_min=5,
    area_km2=None,
    peak_position=0.5  # pico en el centro
)

print(f"Precipitacion total: {hietograma.total_depth_mm:.2f} mm")
print(f"Intensidad pico: {hietograma.peak_intensity_mmhr:.2f} mm/hr")
\end{lstlisting}

\section{Distribuciones SCS}

Las distribuciones SCS (Tipo I, IA, II, III) son curvas de masa adimensionales desarrolladas por el Soil Conservation Service para distintas regiones de Estados Unidos.

\begin{table}[H]
\centering
\caption{Distribuciones SCS y su aplicación}
\label{tab:scs_types}
\begin{tabular}{lll}
\toprule
Tipo & Región & Posición del Pico \\
\midrule
I & Costa del Pacífico & $\sim$10 hr \\
IA & Noroeste del Pacífico & $\sim$8 hr \\
II & Continental (mayoría de EEUU) & 12 hr (50\%) \\
III & Costa del Golfo/Atlántico & $\sim$12 hr \\
\bottomrule
\end{tabular}
\end{table}

\begin{lstlisting}[style=bashstyle, language=bash]
# Distribucion SCS Tipo II, 24 horas, 150mm
python -m hidropluvial storm scs 150 --duration 24 --storm-type II
\end{lstlisting}

\section{Tormentas Bimodales}

Para cuencas urbanas mixtas o tormentas frontales de larga duración, se puede usar una distribución bimodal con dos picos:

\begin{lstlisting}[style=bashstyle, language=bash]
# Tormenta bimodal: 100mm, 6 horas, picos en 25% y 75%
python -m hidropluvial storm bimodal 100 --duration 6 --peak1 0.25 --peak2 0.75
\end{lstlisting}

% ============================================================================
% CAPÍTULO 4: ESCORRENTÍA
% ============================================================================

\chapter{Cálculo de Escorrentía}

\section{Método SCS Curve Number}

\subsection{Ecuación de Escorrentía}

\begin{equation}
Q = \frac{(P - I_a)^2}{P - I_a + S} \quad \text{para } P > I_a
\label{eq:scs_runoff}
\end{equation}

donde:
\begin{itemize}
    \item $Q$: escorrentía directa (mm)
    \item $P$: precipitación total (mm)
    \item $I_a$: abstracción inicial (mm)
    \item $S$: retención potencial máxima (mm)
\end{itemize}

\subsection{Retención Potencial}

\begin{equation}
S = \frac{25400}{CN} - 254 \quad \text{(mm)}
\label{eq:retention}
\end{equation}

\subsection{Abstracción Inicial}

\begin{equation}
I_a = \lambda \times S
\label{eq:ia}
\end{equation}

Tradicionalmente $\lambda = 0{,}20$, aunque estudios recientes sugieren $\lambda = 0{,}05$.

\subsection{Ajuste por Condición de Humedad Antecedente}

\begin{align}
CN_I &= \frac{CN_{II}}{2{,}281 - 0{,}01281 \times CN_{II}} \quad \text{(seco)} \label{eq:cn_i} \\
CN_{III} &= \frac{CN_{II}}{0{,}427 + 0{,}00573 \times CN_{II}} \quad \text{(húmedo)} \label{eq:cn_iii}
\end{align}

\subsection{Uso desde CLI}

\begin{lstlisting}[style=bashstyle, language=bash]
# SCS-CN basico: P=100mm, CN=75
python -m hidropluvial runoff cn 100 75

# Con condicion humeda (AMC III)
python -m hidropluvial runoff cn 100 75 --amc III

# Con lambda=0.05
python -m hidropluvial runoff cn 100 75 --lambda 0.05
\end{lstlisting}

\subsection{Uso desde Python}

\begin{lstlisting}[language=Python]
from hidropluvial.core import calculate_scs_runoff
from hidropluvial.config import AntecedentMoistureCondition

result = calculate_scs_runoff(
    rainfall_mm=100,
    cn=75,
    lambda_coef=0.2,
    amc=AntecedentMoistureCondition.WET  # AMC III
)

print(f"CN ajustado: {result.cn_used}")
print(f"Retencion S: {result.retention_mm:.2f} mm")
print(f"Abstraccion Ia: {result.initial_abstraction_mm:.2f} mm")
print(f"Escorrentia Q: {result.runoff_mm:.2f} mm")
\end{lstlisting}

\section{Método Racional}

\begin{equation}
Q = C_f \times C \times i \times A
\label{eq:rational}
\end{equation}

donde:
\begin{itemize}
    \item $Q$: caudal pico (m³/s)
    \item $C_f$: factor de ajuste por período de retorno
    \item $C$: coeficiente de escorrentía (0-1)
    \item $i$: intensidad de lluvia (mm/hr)
    \item $A$: área de la cuenca (ha)
\end{itemize}

\begin{lstlisting}[style=bashstyle, language=bash]
# Metodo Racional: C=0.6, i=50mm/hr, A=10ha, Tr=25 anos
python -m hidropluvial runoff rational 0.6 50 10 --return-period 25
\end{lstlisting}

% ============================================================================
% CAPÍTULO 5: TIEMPO DE CONCENTRACIÓN
% ============================================================================

\chapter{Tiempo de Concentración}

\section{Fórmula de Kirpich (1940)}

\begin{equation}
t_c = 0{,}0195 \times L^{0{,}77} \times S^{-0{,}385}
\label{eq:kirpich}
\end{equation}

donde:
\begin{itemize}
    \item $t_c$: tiempo de concentración (min)
    \item $L$: longitud del cauce (m)
    \item $S$: pendiente (m/m)
\end{itemize}

\textbf{Factores de ajuste:}
\begin{itemize}
    \item Canales con pasto: $\times 2{,}0$
    \item Concreto/asfalto: $\times 0{,}4$
    \item Canales de concreto: $\times 0{,}2$
\end{itemize}

\begin{lstlisting}[style=bashstyle, language=bash]
# Kirpich: L=2000m, S=0.02
python -m hidropluvial tc kirpich 2000 0.02

# Con superficie de pasto
python -m hidropluvial tc kirpich 2000 0.02 --surface grassy
\end{lstlisting}

\section{Fórmula de Témez}

\begin{equation}
t_c = 0{,}3 \times \left(\frac{L}{S^{0{,}25}}\right)^{0{,}76}
\label{eq:temez}
\end{equation}

donde:
\begin{itemize}
    \item $t_c$: tiempo de concentración (hr)
    \item $L$: longitud del cauce (km)
    \item $S$: pendiente (m/m)
\end{itemize}

Válida para cuencas de 1 a 3000 km².

\begin{lstlisting}[style=bashstyle, language=bash]
# Temez: L=5km, S=0.015
python -m hidropluvial tc temez 5 0.015
\end{lstlisting}

% ============================================================================
% CAPÍTULO 6: HIDROGRAMAS
% ============================================================================

\chapter{Hidrogramas Unitarios}

\section{Hidrograma Triangular SCS}

\subsection{Caudal Pico}

\begin{equation}
q_p = \frac{2{,}08 \times A \times Q}{T_p}
\label{eq:scs_peak}
\end{equation}

donde:
\begin{itemize}
    \item $q_p$: caudal pico (m³/s)
    \item $A$: área de la cuenca (km²)
    \item $Q$: escorrentía (mm)
    \item $T_p$: tiempo al pico (hr)
\end{itemize}

\subsection{Parámetros Temporales}

\begin{align}
t_{lag} &= 0{,}6 \times T_c \label{eq:tlag} \\
T_p &= \frac{\Delta D}{2} + t_{lag} \label{eq:tp} \\
T_r &= 1{,}67 \times T_p \label{eq:tr} \\
T_b &= 2{,}67 \times T_p \label{eq:tb}
\end{align}

\subsection{Uso desde Python}

\begin{lstlisting}[language=Python]
from hidropluvial.core.hydrograph import (
    scs_triangular_uh,
    scs_curvilinear_uh,
    generate_hydrograph
)
from hidropluvial.config import HydrographMethod
import numpy as np

# Hidrograma unitario triangular
time, flow = scs_triangular_uh(
    area_km2=25,
    tc_hr=1.5,
    dt_hr=0.1
)

# Hidrograma unitario curvilinear (PRF=484)
time, flow = scs_curvilinear_uh(
    area_km2=25,
    tc_hr=1.5,
    dt_hr=0.1,
    prf=484
)

# Hidrograma completo con exceso de lluvia
rainfall_excess = np.array([5, 15, 25, 20, 10, 5])  # mm
result = generate_hydrograph(
    rainfall_excess_mm=rainfall_excess,
    method=HydrographMethod.SCS_CURVILINEAR,
    area_km2=25,
    tc_hr=1.5,
    dt_hr=0.1
)

print(f"Caudal pico: {result.peak_flow_m3s:.2f} m3/s")
print(f"Tiempo al pico: {result.time_to_peak_hr:.2f} hr")
\end{lstlisting}

% ============================================================================
% CAPÍTULO 7: GENERACIÓN DE REPORTES
% ============================================================================

\chapter{Generación de Reportes}

\section{Reportes LaTeX}

\hidropluvial{} puede generar memorias de cálculo completas en formato \LaTeX, listas para compilar a PDF.

\subsection{Reporte de Curvas IDF}

\begin{lstlisting}[style=bashstyle, language=bash]
python -m hidropluvial report idf 78 -o informe_idf.tex --author "Ing. Garcia"
\end{lstlisting}

El reporte incluye:
\begin{itemize}
    \item Datos de entrada
    \item Tabla de intensidades para todos los períodos de retorno
    \item Fórmulas utilizadas
    \item Metadatos del documento
\end{itemize}

\subsection{Reporte de Hietograma}

\begin{lstlisting}[style=bashstyle, language=bash]
python -m hidropluvial report storm 78 3 --tr 25 -o informe_storm.tex
\end{lstlisting}

El reporte incluye:
\begin{itemize}
    \item Datos de entrada
    \item Resumen de resultados
    \item Gráfico TikZ del hietograma
    \item Tabla de datos tabulados
\end{itemize}

\section{Exportación de Datos}

\subsection{CSV}

\begin{lstlisting}[style=bashstyle, language=bash]
# Tabla IDF
python -m hidropluvial export idf-csv 78 -o tabla_idf.csv

# Hietograma
python -m hidropluvial export storm-csv 78 3 --tr 25 -o hietograma.csv
\end{lstlisting}

\subsection{Figuras TikZ}

Para incluir figuras en un documento \LaTeX{} existente:

\begin{lstlisting}[style=bashstyle, language=bash]
python -m hidropluvial export storm-tikz 78 3 --tr 25 -o figura.tex
\end{lstlisting}

Luego en el documento principal:

\begin{lstlisting}[language=TeX]
\input{figura.tex}
\end{lstlisting}

\section{Generación desde Python}

\begin{lstlisting}[language=Python]
from hidropluvial.core import alternating_blocks_dinagua
from hidropluvial.reports import (
    ReportGenerator,
    hyetograph_result_to_tikz
)

# Generar hietograma
hietograma = alternating_blocks_dinagua(78, 25, 3, 5)

# Convertir a TikZ
tikz = hyetograph_result_to_tikz(
    hietograma,
    caption="Hietograma Tr=25 anos",
    label="fig:hietograma"
)

# Guardar
with open("hietograma.tex", "w") as f:
    f.write(tikz)

# Generar documento completo
generator = ReportGenerator()
doc = generator.generate_standalone_document(
    content=tikz,
    title="Memoria de Calculo",
    author="Ing. Garcia"
)

with open("memoria.tex", "w") as f:
    f.write(doc)
\end{lstlisting}

% ============================================================================
% CAPÍTULO 8: REFERENCIA DE COMANDOS
% ============================================================================

\chapter{Referencia de Comandos CLI}

\section{Comando Principal}

\begin{lstlisting}[style=bashstyle, language=bash]
python -m hidropluvial [OPTIONS] COMMAND [ARGS]

Options:
  --version, -v    Mostrar version
  --help           Mostrar ayuda
\end{lstlisting}

\section{Subcomandos}

\subsection{idf}

\begin{tabular}{ll}
\toprule
Comando & Descripción \\
\midrule
\comando{idf uruguay <P3\_10> <dur> --tr <T>} & Intensidad DINAGUA \\
\comando{idf tabla-uy <P3\_10>} & Tabla IDF completa \\
\comando{idf departamentos} & Valores $P_{3,10}$ \\
\comando{idf sherman <dur> <T>} & Intensidad Sherman \\
\bottomrule
\end{tabular}

\subsection{storm}

\begin{tabular}{ll}
\toprule
Comando & Descripción \\
\midrule
\comando{storm uruguay <P3\_10> <dur>} & Hietograma DINAGUA \\
\comando{storm bimodal <depth>} & Tormenta bimodal \\
\comando{storm scs <depth>} & Distribución SCS \\
\comando{storm chicago <depth>} & Método Chicago \\
\bottomrule
\end{tabular}

\subsection{tc}

\begin{tabular}{ll}
\toprule
Comando & Descripción \\
\midrule
\comando{tc kirpich <L> <S>} & Fórmula Kirpich \\
\comando{tc temez <L> <S>} & Fórmula Témez \\
\bottomrule
\end{tabular}

\subsection{runoff}

\begin{tabular}{ll}
\toprule
Comando & Descripción \\
\midrule
\comando{runoff cn <P> <CN>} & Método SCS-CN \\
\comando{runoff rational <C> <i> <A>} & Método Racional \\
\bottomrule
\end{tabular}

\subsection{report}

\begin{tabular}{ll}
\toprule
Comando & Descripción \\
\midrule
\comando{report idf <P3\_10>} & Memoria IDF \\
\comando{report storm <P3\_10> <dur>} & Memoria hietograma \\
\bottomrule
\end{tabular}

\subsection{export}

\begin{tabular}{ll}
\toprule
Comando & Descripción \\
\midrule
\comando{export idf-csv <P3\_10>} & Tabla IDF a CSV \\
\comando{export storm-csv <P3\_10> <dur>} & Hietograma a CSV \\
\comando{export storm-tikz <P3\_10> <dur>} & Figura TikZ \\
\bottomrule
\end{tabular}

% ============================================================================
% APÉNDICES
% ============================================================================

\appendix

\chapter{Tablas de Referencia}

\section{Números de Curva (CN)}

\begin{table}[H]
\centering
\caption{Números de curva para áreas urbanas (TR-55)}
\label{tab:cn_urban}
\begin{tabular}{lcccc}
\toprule
Descripción & HSG A & HSG B & HSG C & HSG D \\
\midrule
Espacio abierto (buena condición) & 39 & 61 & 74 & 80 \\
Áreas impermeables & 98 & 98 & 98 & 98 \\
Residencial 1/4 acre & 61 & 75 & 83 & 87 \\
Comercial (85\% impermeable) & 89 & 92 & 94 & 95 \\
Industrial & 81 & 88 & 91 & 93 \\
Calles pavimentadas & 98 & 98 & 98 & 98 \\
\bottomrule
\end{tabular}
\end{table}

\begin{table}[H]
\centering
\caption{Números de curva para áreas agrícolas (TR-55)}
\label{tab:cn_agri}
\begin{tabular}{lcccc}
\toprule
Descripción & HSG A & HSG B & HSG C & HSG D \\
\midrule
Barbecho & 77 & 86 & 91 & 94 \\
Cultivos en hilera (buena condición) & 67 & 78 & 85 & 89 \\
Pasturas (buena condición) & 39 & 61 & 74 & 80 \\
Praderas & 30 & 58 & 71 & 78 \\
Bosques (buena condición) & 30 & 55 & 70 & 77 \\
\bottomrule
\end{tabular}
\end{table}

\section{Coeficientes de Escorrentía (C)}

\begin{table}[H]
\centering
\caption{Coeficientes de escorrentía para método racional}
\label{tab:c_rational}
\begin{tabular}{lc}
\toprule
Uso del suelo & C \\
\midrule
Centro comercial & 0,70 -- 0,95 \\
Residencial unifamiliar & 0,30 -- 0,50 \\
Industrial ligero & 0,50 -- 0,80 \\
Asfalto/concreto & 0,70 -- 0,95 \\
Parques/espacios abiertos & 0,10 -- 0,25 \\
\bottomrule
\end{tabular}
\end{table}

\chapter{Referencias Bibliográficas}

\begin{enumerate}
    \item Rodríguez Fontal, A. (1980). \textit{Regionalización de lluvias máximas en Uruguay}.

    \item DINAGUA/MTOP. \textit{Manual de Drenaje Pluvial Urbano}.

    \item USDA-NRCS (1986). \textit{Urban Hydrology for Small Watersheds}. Technical Release 55 (TR-55).

    \item FHWA (2023). \textit{Highway Hydrology}. Hydraulic Design Series No. 2 (HDS-2), 3rd Edition.

    \item Chow, V.T., Maidment, D.R., Mays, L.W. (1988). \textit{Applied Hydrology}. McGraw-Hill.

    \item Huff, F.A. (1967). Time Distribution of Rainfall in Heavy Storms. \textit{Water Resources Research}, 3(4):1007-1019.

    \item Keifer, C.J., Chu, H.H. (1957). Synthetic Storm Pattern for Drainage Design. \textit{Journal of the Hydraulics Division, ASCE}, 83(HY4):1-25.
\end{enumerate}

% ============================================================================
% FIN DEL DOCUMENTO
% ============================================================================

\end{document}
