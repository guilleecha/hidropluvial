\documentclass[11pt, a4paper]{article}

% Paquetes básicos
\usepackage[utf8]{inputenc}
\usepackage[spanish]{babel}
\usepackage[margin=2.5cm]{geometry}
\usepackage{booktabs}
\usepackage{siunitx}
\usepackage{amsmath}
\usepackage{float}
\usepackage{graphicx}
\usepackage{hyperref}
\usepackage{tikz}
\usepackage{pgfplots}
\pgfplotsset{compat=1.18}

% Metadatos
\title{Memoria de Cálculo: Hietograma de Diseño}
\author{\VAR{author | escape_latex}}
\date{\VAR{date}}

\begin{document}

\maketitle

\section{Datos de Entrada}

\begin{table}[H]
\centering
\begin{tabular}{lr}
\toprule
Parámetro & Valor \\
\midrule
Método & \VAR{method} \\
$P_{3,10}$ base & \VAR{p3_10} mm \\
Período de retorno & \VAR{return_period} años \\
Duración de tormenta & \VAR{duration} horas \\
Intervalo $\Delta t$ & \VAR{dt} minutos \\
\BLOCK{if area}
Área de cuenca & \VAR{area} km² \\
\BLOCK{endif}
\bottomrule
\end{tabular}
\caption{Parámetros de entrada}
\label{tab:input}
\end{table}

\section{Resultados}

\begin{table}[H]
\centering
\begin{tabular}{lr}
\toprule
Resultado & Valor \\
\midrule
Precipitación total & \VAR{total_depth} mm \\
Intensidad pico & \VAR{peak_intensity} mm/hr \\
Número de intervalos & \VAR{n_intervals} \\
\bottomrule
\end{tabular}
\caption{Resumen de resultados}
\label{tab:results}
\end{table}

\section{Hietograma}

\VAR{hyetograph_figure}

\section{Datos Tabulados}

\VAR{hyetograph_table}

\vfill
\noindent\rule{\textwidth}{0.4pt}
\footnotesize
Generado por HidroPluvial v\VAR{version} el \VAR{date}.

\end{document}
