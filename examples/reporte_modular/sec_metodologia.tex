% Seccion: Metodologia
% Generado automaticamente por HidroPluvial

\section{Metodología}

\subsection{Tiempo de Concentración}

\textbf{Método de Kirpich:}
\begin{equation}
T_c = 0.0195 \times L^{0.77} \times S^{-0.385}
\end{equation}

donde $L$ es la longitud del cauce principal en metros y $S$ la pendiente media (m/m).

\textbf{Método de los Desbordes (DINAGUA):}
\begin{equation}
T_c = T_0 + 6.625 \times A^{0.3} \times P^{-0.39} \times C^{-0.45}
\end{equation}

donde $A$ es el área en hectáreas, $P$ la pendiente en \%, $C$ el coeficiente de escorrentía, y $T_0 = 5$ min.

\subsection{Curvas IDF DINAGUA}

Factor por período de retorno:
\begin{equation}
C_T(T_r) = 0.5786 - 0.4312 \times \log_{10}\left[\ln\left(\frac{T_r}{T_r - 1}\right)\right]
\end{equation}

Ecuaciones de intensidad:
\begin{equation}
I(d) = \frac{P_{3,10} \times C_T(T_r) \times 0.6208}{(d + 0.0137)^{0.5639}} \quad \text{para } d < 3 \text{ horas}
\end{equation}

\begin{equation}
I(d) = \frac{P_{3,10} \times C_T(T_r) \times 1.0287}{(d + 1.0293)^{0.8083}} \quad \text{para } d \geq 3 \text{ horas}
\end{equation}

\subsection{Hidrograma Unitario Triangular}

\begin{equation}
T_p = 0.5 \times \Delta t + 0.6 \times T_c
\end{equation}

\begin{equation}
q_p = 0.278 \times \frac{A}{T_p} \times \frac{2}{1 + X}
\end{equation}

\begin{equation}
T_b = (1 + X) \times T_p
\end{equation}

donde $A$ es el área en km², $T_p$ el tiempo al pico en horas, y $X$ el factor morfológico.
